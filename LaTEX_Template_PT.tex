\documentclass[12pt,reqno, twoside]{amsbook}
%%%%%%%%%%%%%%%%%%%%%%%%%%%%%%%%%%%%%%%%%%%%%%%%%%%%%%%%%%%%%%%%%%%%%%%%%%%%%%%%%%%%%%%%%%%%%%%%%%%%%%%%%%%%%%%%%%%%%%%%%%%%%%%%%%%%%%%%%%%%%%%%%%%%%%%%%%%%%%%%%%%%%%%%%%%%%%%%%%%%%%%%%%%%%%%%%%%%%%%%%%%%%%%%%%%%%%%%%%%%%%%%%%%%%%%%%%%%%%%%%%%%%%%%%%%%
\usepackage{eurosym}
\usepackage{amsmath}
\usepackage{amssymb}
\usepackage{amsfonts}
\usepackage[onehalfspacing]{setspace}
\usepackage{chngcntr}
\usepackage{graphicx}
\usepackage[a4paper, margin=2.5cm]{geometry}
\usepackage[portuges]{babel}
\usepackage{fancyhdr}
\usepackage{titlesec}
\usepackage{enumitem}
\usepackage{etoolbox}
% pode ter de retirar em baixo a diretiva de commentario quando usar figuras em formato eps
%\usepackage{epstopdf}


\makeatletter
    \def\section{\@startsection{section}{1}%
      \z@{.5\linespacing\@plus.7\linespacing}{.25\linespacing}%
      {\normalfont\bfseries\flushleft}}
    \def\subsection{\@startsection{subsection}{2}%
      \z@{.5\linespacing\@plus.7\linespacing}{.25\linespacing}%
      {\normalfont\bfseries\flushleft}}

\makeatother
\setcounter{MaxMatrixCols}{10}

\providecommand{\U}[1]{\protect\rule{.1in}{.1in}}
\theoremstyle{plain}
\newtheorem{acknowledgement}{Agradecimento}
\newtheorem{algorithm}{Algoritmo}[chapter]
\newtheorem{axiom}{Axioma}[chapter]
\newtheorem{case}{Caso}[chapter]
\newtheorem{claim}{Claim}[chapter]
\newtheorem{conclusion}{Conclus\~{a}o}[chapter]
\newtheorem{condition}{Condi\c{c}\~{a}o}[chapter]
\newtheorem{conjecture}{Conjetura}[chapter]
\newtheorem{corollary}{Corol\'{a}rio}[chapter]
\newtheorem{criterion}{Crit\'{e}rio}[chapter]
\newtheorem{definition}{Defini\c{c}\~{a}o}[chapter]
\newtheorem{example}{Exemplo}[chapter]
\newtheorem{exercise}{Exerc\'{i}cio}[chapter]
\newtheorem{lemma}{Lema}[chapter]
\newtheorem{notation}{Nota\c{c}\~{a}o}[chapter]
\newtheorem{problem}{Problema}[chapter]
\newtheorem{proposition}{Proposi\c{c}\~{a}o}[chapter]
\newtheorem{remark}{Nota}[chapter]
\newtheorem{solution}{Solu\c{c}\~{a}o}[chapter]
\newtheorem{summary}{Sum\'{a}rio}[chapter]
\newtheorem{theorem}{Teorema}[chapter]
\numberwithin{equation}{chapter}


% Please write the references according to your school

\newenvironment{dedication}
  {%\clearpage           % we want a new page
   \thispagestyle{empty}% no header and footer
   \vspace*{\stretch{1}}% some space at the top
   \itshape             % the text is in italics
   \raggedleft          % flush to the right margin
  }
  {\par % end the paragraph
   \vspace{\stretch{3}} % space at bottom is three times that at the top
   %\clearpage           % finish off the page
  }
  \numberwithin{section}{chapter}
\fancyhead{}
\fancyfoot{}
\pagestyle{fancy}
\fancyfoot[LE,RO]{\thepage}

\makeatletter
\def\ps@plain{\ps@empty
 \def\@evenfoot{%
   \normalfont\scriptsize
   \rlap{\thepage}\hfil
   }%
 \def\@oddfoot{%
   \normalfont\scriptsize \hfil
   \llap{\thepage}}%
}
\makeatother
\renewcommand{\headrulewidth}{0pt}
\renewcommand{\footrulewidth}{0pt}



\begin{document}
\frontmatter
\addtocontents{toc}{\setcounter{tocdepth}{2}}
\thispagestyle{empty}

\begin{dedication}%

\begin{flushright}
\textit{Escreva aqui a sua dedicat\'{o}ria}
\end{flushright}%
\end{dedication}

\setcounter{page}{1}
\pagenumbering{roman} %
\chapter*{Agradecimento}

Escreva aqui os agradecimentos e, caso haja, as fontes de financiamento

\chapter*{Resumo}

Escreva aqui o resumo em portugu\^{e}s

\chapter*{Abstract}

Escreva aqui o resumo em ingl\^{e}s

\renewcommand{\contentsname}{\'{I}ndice}
\tableofcontents

\mainmatter

\hyphenation{wri-te he-re pro-per hy-phe-ni-za-tion
}%

\setcounter{page}{1} \pagenumbering{arabic}

\chapter{Introdu\c{c}\~{a}o}
\noindent Deve usar o comando $\backslash$noindent no primeiro par\'{a}grafo de cada sec\c{c}\~{a}o e subsec\c{c}\~{a}o.
\section{Sec\c{c}\~{a}o Maior}
\noindent Use o comando $\backslash$section para iniciar uma sec\c{c}\~{a}o.
\subsection{Esta \'{e} uma subsec\c{c}\~{a}o}
\noindent Use o comando $\backslash$subsection para iniciar uma subsec\c{c}\~{a}o..
\subsection{Esta \'{e} uma outra subsec\c{c}\~{a}o}
\noindent \'{E} um texto sem qualquer significado.\par
Aqui come\c{c}a o segundo par\'{a}grafo.
\section{Outra Sec\c{c}\~{a}o Maior}
\newpage
Adicionamos uma p\'{a}gina para verificar que o n´\'{u}mero da  p\'{a}gina par fica do lado esquerdo.

\chapter{Revis\~{a}o da Literatura}
\section{Uma Sec\c{c}\~{a}o}
\section{Outra Sec\c{c}\~{a}o}


\chapter{Aspetos Matem\'{a}ticos}
\section{Matem\'{a}tica em Texto}
\noindent Seja $H$ um espa\c{c}o um Euclideano e $C$ um subconjunto convexo de $H$, ...
Suponha que quando $n\rightarrow\infty$, ....

\section{F\'{o}rmulas Matem\'{a}ticas}
\noindent Exemplifica-se como as equa\c{c}\~{o}es podem ficar numeradas e como o seu n\'{u}mero pode ser invocado pela etiqueta correspodente.
\begin{equation}
w_{tt}-\Delta w+w^{6}+w\left|  w\right|  ^{p-2}=0\text{ in }\mathbf{R}%
^{3}\times\left[  0,\infty\right) \label{eqn1}%
\end{equation}

A equa\c{c}\~{a}o (\ref{eqn1}) mostra que

\chapter{Contextos do Tipo Teorema}
\section{Alguns Exemplos}

\begin{algorithm}
Isto \'{e} um algoritmo.
\end{algorithm}

\begin{algorithm}
Isto \'{e} outro algoritmo.
\end{algorithm}

\begin{conjecture}
Isto \'{e} uma conjetura
\end{conjecture}

\begin{corollary}
Isto \'{e} um corol\'{a}rio.
\end{corollary}

\begin{corollary}
Isto \'{e} outro corol\'{a}rio.
\end{corollary}

\begin{corollary}
Ainda mais um corol\'{a}rio.
\end{corollary}

\begin{criterion}
Isto \'{e} um crit\'{e}rio.
\end{criterion}

\begin{definition}
Isto \'{e} uma defini\c{c}\~{a}o.
\end{definition}

\begin{example}
Isto \'{e} um exemplo.
\end{example}

\begin{exercise}
Isto \'{e} um exerc\'{i}cio.
\end{exercise}

\begin{lemma}
Isto \'{e} um lema.
\end{lemma}

\begin{proof}
Isto \'{e} uma prova do lema.
\end{proof}

\begin{notation}
Isto \'{e} uma nota\c{c}\~{a}o.
\end{notation}

\begin{problem}
Isto \'{e} um problema.
\end{problem}

\begin{proposition}
Isto \'{e} uma proposi\c{c}\~{a}o.
\end{proposition}


\begin{proof}
[Prova do Teorema Principal] Faz-se aqui a prova .
\end{proof}
\chapter{Conclus\~{o}es}

\renewcommand{\bibname}{Refer\^{e}ncias Bibliogr\'{a}ficas}


\def\bibindent{0.7cm}
\begin{thebibliography}{99\kern\bibindent}
\makeatletter
\let\old@biblabel\@biblabel
\def\@biblabel#1{\old@biblabel{#1}\kern\bibindent}
\let\old@bibitem\bibitem
\def\bibitem#1{\old@bibitem{#1}\leavevmode\kern-\bibindent}
\makeatother
\makeatletter
\renewcommand\@biblabel[1]{}
\makeatother

%\begin{thebibliography}{99}
\bibitem{aka73} H. Akaike (1973), \textquotedblleft Information Theory as an
Extension of the Maximum Likelihood Principle\textquotedblright, in B. N.
Petrov, and F. Csaki, (Eds.), \textit{Second International Symposium on
Information Theory}, Akademiai Kiado, Budapest, pp. 267--281.

\bibitem{and10} D.T. Anderson, J.C. Bezdek, M. Popescu, and J.M. Keller
(2010), \textquotedblleft Comparing Fuzzy, Probabilistic, and Possibilistic
Partitions\textquotedblright, \textit{IEEE Transactions on Fuzzy Systems},
18(5), 906--918.
\end{thebibliography}

\end{document}
